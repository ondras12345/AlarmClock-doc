\subsection{Displej}
Aby byl budík uživatelsky přívětivý, je nutné umožnit jeho nastavování bez
nutnosti připojení jiného zařízení (počítače, mobilního telefonu apod.).
Pro zobrazování dat se v dnešní době využívá široká škála různých technologií
displejů. Vzhledem k textovému charakteru zobrazovaných dat jsou pro toto
zařízení vhodné znakové či grafické displeje, segmentové displeje by neumožnily
zobrazení většího množství údajů na jednom místě. Protože není potřeba
zobrazovat složitou grafiku, zvolil jsem znakový LCD displej (16 znaků na
řádek, 2 řádky) s řadičem HD44780. Ten je asi nejlevnější alternativou, je
snadno dostupný a díky možnosti nadefinovat několik vlastních znaků lze
zobrazovat i jednoduchou grafiku.

Řadič HD44780 využívá pro komunikaci 8bitovou sběrnici ($\mathrm{DB0}$ až
$\mathrm{DB7}$) a tři řídicí piny: $\mathrm{R}/\overline{\mathrm{W}}$,
$\mathrm{E}$ a $\mathrm{RS}$. Pin $\mathrm{R}/\overline{\mathrm{W}}$ určuje,
jestli požaduje mikrokontrolér čtení dat z řadiče (logická 1) či zápis (logická
0). V obou případech se data přenášejí po sběrnici. Pin $\mathrm{E}$ zahajuje
operaci čtení či zápisu. $\mathrm{RS}$ určuje, jestli jsou data na sběrnici
daty pro zápis do paměti řadiče ($\mathrm{RS} = 1$, obvykle zápis ASCII hodnoty
požadovaného znaku) nebo instrukce, kterou má řadič provést ($\mathrm{RS} =
0$). Podporované instrukce zahrnují smazání displeje, zapnutí blikání kurzoru
apod. Řadič umožňuje i provoz ve 4bitovém režimu, v takovém případě se ze
sběrnice využijí pouze piny $\mathrm{DB4}$ až $\mathrm{DB7}$.
% TODO citovat datasheet HD44780

% TODO
