\section{PlatformIO}
PlatformIO\footnote{\url{https://platformio.org/}} je multiplatformní, na
architektuře nezávislý nástroj s otevřeným zdrojovým kódem pro vývoj embedded
systémů. Podporuje mnoho SDK (včetně Arduino) a nezávisí na konkrétním
vývojovém prostředí (IDE). Oproti Arduino IDE nabízí možnost pracovat
s příkazovým řádkem namísto grafického rozhraní a díky tomu umožňuje
jednoduchou automatizaci různých procesů ve vývoji. Konfigurační soubor
\filename{platformio.ini} také umožňuje nastavit všechny parametry kompilace,
včetně přímého předávání přepínačů kompilátoru \shellcmd{gcc}. Je možné
vytvořit projekt, který je kompatibilní s Arduino IDE i s PlatformIO, ale to
pouze ztěžuje využívání pokročilých funkcí PlatformIO a nepřináší žádné výhody.

Nespornou výhodou PlatformIO je také správa závislostí -- knihoven, bez kterých
není možné firmware zkompilovat. Tím, že v konfiguračním souboru
\filename{platformio.ini} zaznamenává jejich přesné verze, odstraňuje potřebu
vytvářet tuto dokumentaci ručně. Znalost verzí knihoven, se kterými byl
firmware testován, je kritická pro budoucí kompilování. Může se totiž stát, že
v knihovně dojde v novější verzi ke změně API či zanesení chyby. Přístup
Arduino IDE (knihovny sdílené mezi projekty, notifikace doporučující
aktualizaci kdykoli je dostupná novější verze) není pro složitější projekty
závisející na mnoha knihovnách vhodný.

Ukázka obsahu konfiguračního souboru \filename{platformio.ini}:
\begin{lstlisting}[language=Ini]
; PlatformIO Project Configuration File
;
;   Build options: build flags, source filter
;   Upload options: custom upload port, speed and extra flags
;   Library options: dependencies, extra library storages
;   Advanced options: extra scripting
;
; Please visit documentation for the other options and examples
; https://docs.platformio.org/page/projectconf.html

[platformio]
src_dir = AlarmClock

[env:uno]
platform = atmelavr
board = uno
framework = arduino
lib_deps = 
	adafruit/RTClib@^2.0.2
	thomasfredericks/Bounce2@^2.70.0
	paulstoffregen/Encoder@^1.4.1
	marcoschwartz/LiquidCrystal_I2C@^1.1.4
	paulstoffregen/TimerOne@^1.1.0
\end{lstlisting}
Skutečný konfigurační soubor je složitější, protože obsahuje i nastavení
jednotkových testů a různé další úpravy.

Práce s nástrojem pro příkazový řádek PlatformIO Core (CLI) je velmi
jednoduchá. Zkompilování firmware se provádí příkazem
\shellcmd{pio run --environment uno}. Přepínač \texttt{--environment} určuje,
pro jakou platformu se má firmware zkompilovat. Lze použít jeho krátkou verzi
\texttt{-e}. Výsledný soubor ve formátu Intel HEX se poté nachází v umístění
\repopath{.pio/build/uno/firmware.hex}. Smazání všech zkompilovaných souborů
(užitečné pro vynucení úplného překompilování) se provádí příkazem
\shellcmd{pio run -e uno -t clean}, kde \texttt{-t} je krátká verze přepínače
\texttt{--target}. Nahrání firmware do mikrokontroléru provádíme příkazem
\shellcmd{pio run -e uno -t upload}.

Pro další zjednodušení a automatizaci práce s tímto nástrojem byl vytvořen
\filename{Makefile}, tedy soubor s pravidly programu GNU \shellcmd{make}.
Tento soubor obsahuje i mechanismus pro generování nápovědy, dostupné příkazy
lze zjistit příkazem \shellcmd{make help}.

\subsection{Automatické testy}
Jednou z hlavních motivací pro nasazení technologie PlatformIO byla podpora
automatických testů. PlatformIO podporuje testování jednotlivých jednotek
zdrojového kódu (unit testing), a to jak na cílové platformě (embedded MCU),
tak i na počítači použitém pro vývoj (native). Testy jsou umístěny
v samostatných podsložkách v adresáři \repopath{test/}.

Pro demonstraci využití jednotkových testů je vhodná velmi jednoduchá jednotka
zajišťující ukládání boolean hodnoty (jednoho bitu -- pravda či nepravda) pro
každý den v týdnu -- třída \texttt{DaysOfWeek}. Celý stav tohoto objektu je
reprezentován jedním osmibitovým číslem. Protože je toto číslo přenášeno
rozhraním UART z mikrokontroléru k připojenému počítači, je potřeba zajistit,
že obě strany ukládají hodnoty stejným způsobem. Manuální testování není
vhodné, protože potřebujeme zabránit možnosti náhodného neúmyslného poškození
funkcionality při pozdějších úpravách kódu. Spouštění automatických testů při
každé změně zdrojového kódu totiž zabírá několikanásobně méně času než manuální
testování. Jednotkové testy navíc odstraňují možnost lidské chyby.

Pro pochopení konceptu jednotkových testů nepotřebujeme znát vnitřní
implementaci třídy \texttt{DaysOfWeek} a postačí nám pouhé deklarace
metod z hlavičkového souboru:
\begin{lstlisting}[language=C++]
#include <stdint.h>

class DaysOfWeek
{
public:
    uint8_t days_of_week = 0x00;
    bool getDayOfWeek(uint8_t num) const;
    bool setDayOfWeek(uint8_t num, bool status);
};
\end{lstlisting}

Zdrojový kód testu umístíme do souboru
\repopath{test/test_DaysOfWeek/test_DaysOfWeek.cpp}. Připravíme si několik
případů, pro které chceme chování jednotky testovat. Není nezbytně nutné
otestovat všechny možné stavy a často to ani není prakticky možné. Obvykle
postačí několik běžných stavů a případy, které jsou nějakým způsobem speciální
a je u nich proto větší riziko chyby.
\begin{lstlisting}[language=C++,style=numbers,breakatwhitespace=true]
#include <unity.h>  // unit testing framework
#include <stdint.h>
#include <DaysOfWeek.h>  // testovaná jednotka


struct DowTestCase
{
    uint8_t code;
    bool days[7];
};

// Pole testovaných případů:
DowTestCase Cases[] = {
    {0x00, {false,  false,  false,  false,  false,  false,  false}},
    {0x01, {false,  false,  false,  false,  false,  false,  false}},
    {0x02, {true,   false,  false,  false,  false,  false,  false}},
    // ...
    {0xFE, {true,   true,   true,   true,   true,   true,   true}}
};


// Test dekódování 8bitového čísla na jednotlivé dny v týdnu:
void test_decode()
{
    // Pro každý testovaný případ...
    for (size_t i = 0; i < sizeof(Cases)/sizeof(Cases[0]); i++)
    {
        // ...vytvoříme nový objekt DaysOfWeek...
        DaysOfWeek dow;
        // ...a přiřadíme mu dané 8bitové číslo.
        dow.days_of_week = Cases[i].code;
        // V rámci daného případu iterujeme přes všechny dny v týdnu...
        for (size_t j = 0; j < sizeof(Cases[i].days)/sizeof(Cases[i].days[0]); j++)
        {
            // ...a porovnáváme správnou hodnotu daného dne
            // s hodnotou získanou z objektu dow.
            if (Cases[i].days[j])
            {
                TEST_ASSERT_TRUE(dow.getDayOfWeek(j+1));
            }
            else
            {
                TEST_ASSERT_FALSE(dow.getDayOfWeek(j+1));
            }
        }
    }
}


// Test kódování jednotlivých dnů v týdnu do 8bitového čísla:
void test_encode()
{
    // Pro každý testovaný případ...
    for (size_t i = 0; i < sizeof(Cases)/sizeof(Cases[0]); i++)
    {
        // Ošetříme zvláštní případ, který souvisí s faktem,
        // že nejméně významný bit (LSB) je nepoužitý a vždy nulový.
        uint8_t code = Cases[i].code;
        if (code == 0x01) code = 0x00;
        // ...vytvoříme nový objekt DaysOfWeek...
        DaysOfWeek dow;
        // ...a natavíme mu hodnoty jednotlivých dnů v týdnu.
        for (size_t j = 0; j < sizeof(Cases[i].days)/sizeof(Cases[i].days[0]); j++)
        {
            dow.setDayOfWeek(j+1, Cases[i].days[j]);
        }
        // Zkontrojume, zda se 8bitové číslo specifikované testovaným případem
        // shoduje s číslem vytvořeným naším objektem.
        TEST_ASSERT_EQUAL_INT(code, dow.days_of_week);
    }
}


int main(int argc, char **argv)
{
    UNITY_BEGIN();
    // Spustíme výše definované testy.
    RUN_TEST(test_decode);
    RUN_TEST(test_encode);
    UNITY_END();

    return 0;
}
\end{lstlisting}

Spuštění testů provádíme příkazem \shellcmd{pio test}. PlatformIO testovací
program automaticky zkompiluje, nahraje do mikrokontroléru a přes UART přečte
výsledky testů.
% -f test_DaysOfWeek, aby nespoustel ostatni testy
% fixed columns, aby byly zarovnane vysledky
% delka #### progress baru upravena, aby vysly na jeden radek
\begin{lstlisting}[style=terminal,columns=fixed]
$ pio test -e uno
Verbose mode can be enabled via `-v, --verbose` option
Collected 2 items

Processing test_DaysOfWeek in uno environment
-------------------------------------------------------------------
Building...
Uploading...

Warning! Please install `99-platformio-udev.rules`. 
More details: https://docs.platformio.org/page/faq.html#platformio-udev-rules


avrdude: AVR device initialized and ready to accept instructions

Reading | ############################################ | 100% 0.00s

avrdude: Device signature = 0x1e950f (probably m328p)
avrdude: reading input file ".pio/build/uno/firmware.hex"
avrdude: writing flash (3600 bytes):

Writing | ############################################ | 100% 0.60s

avrdude: 3600 bytes of flash written
avrdude: verifying flash memory against .pio/build/uno/firmware.hex:
avrdude: load data flash data from input file .pio/build/uno/firmware.hex:
avrdude: input file .pio/build/uno/firmware.hex contains 3600 bytes
avrdude: reading on-chip flash data:

Reading | ############################################ | 100% 0.47s

avrdude: verifying ...
avrdude: 3600 bytes of flash verified

avrdude: safemode: Fuses OK (E:00, H:00, L:00)

avrdude done.  Thank you.

Testing...
If you don't see any output for the first 10 secs, please reset board (press reset button)

test/test_DaysOfWeek/test_DaysOfWeek.cpp:70:test_decode	[PASSED]
test/test_DaysOfWeek/test_DaysOfWeek.cpp:71:test_encode	[PASSED]
-----------------------
2 Tests 0 Failures 0 Ignored
=================== [PASSED] Took 6.00 seconds ===================

Test             Environment    Status    Duration
---------------  -------------  --------  ------------
test_DaysOfWeek  uno            PASSED    00:00:06.002
=================== 1 succeeded in 00:00:06.002 ===================
\end{lstlisting}

Protože jednotka \texttt{DaysOfWeek} nezávisí na funkcích specifických pro
mikrokontrolér, lze tento test spouštět i v prostředí \texttt{native}, tedy
přímo na PC, na kterém se používá PlatformIO:
\begin{lstlisting}[style=terminal,columns=fixed]
$ pio test -e native
Verbose mode can be enabled via `-v, --verbose` option
Collected 2 items

Processing test_DaysOfWeek in native environment
----------------------------------------------------------------
Building...
Testing...
test/test_DaysOfWeek/test_DaysOfWeek.cpp:70:test_decode	[PASSED]
test/test_DaysOfWeek/test_DaysOfWeek.cpp:71:test_encode	[PASSED]

-----------------------
2 Tests 0 Failures 0 Ignored
OK
================== [PASSED] Took 0.42 seconds ==================

Test             Environment    Status    Duration
---------------  -------------  --------  ------------
test_DaysOfWeek  native         PASSED    00:00:00.419
================= 1 succeeded in 00:00:00.419 =================
\end{lstlisting}

Na druhé straně komunikačního rozhraní UART jsou využitý podobné testy,
tentokrát v programovacím jazyce Python, které kontrolují stejné případy
v tamní implementaci objektu.


\subsubsection{Pomocné objekty -- mock objects}
Abychom mohli testovat i složitější jednotky odděleně od ostatních, využíváme
pomocné objekty. Při testování firmware je například velmi výhodné takto
nahradit kód komunikující s hardware, protože tím je umožněno spouštění testů
v prostředí \texttt{native}, naprosto odděleně od vlastního hardware.

Knihovna
ArduinoFake\footnote{\url{https://github.com/FabioBatSilva/ArduinoFake}}
poskytuje předdefinované pomocné objekty nahrazující funkce specifické pro
Arduino -- například \texttt{Serial}.
% TODO mocking


\subsection{Statická analýza kódu}
Statická analýza kódu umožňuje automaticky detekovat potenciální chyby bez
spuštění programu. Program provádějící statickou analýzu čte zdrojové kódy
a odhaluje chyby jako indexování pole mimo jeho hranice, dereference nulového
ukazatele, neinicializovaná proměnná a podobné.

PlatformIO umožňuje využít více různých nástrojů pro statickou analýzu, přičemž
pro jejich obsluhu zavádí příkaz \shellcmd{pio check}. Snadno použitelný
nástroj je \shellcmd{cppcheck}, kterým se tato práce bude dále zabývat.

Základní nastavení je velmi jednoduché. Provádí se přidáním záznamu do souboru
\filename{platformio.ini}:
\begin{lstlisting}[language=Ini]
[env:uno]
; ...
check_tool = cppcheck
check_skip_packages = true
check_flags = 
	cppcheck: --suppress=unusedFunction --inline-suppr
check_patterns = 
	AlarmClock/
	lib/
	test/
\end{lstlisting}
Volba \inikey{check_skip_packages} určuje, zda se mají při analýze přeskočit
balíčky třetích stran, jako například framework \texttt{arduino}. Při jejich
kontrole totiž může dojít vlivem nedokonalosti nástroje \shellcmd{cppcheck}
a jeho nekompatibility s atributem \lstinline[language={[GNU]C++}]!__asm__!
(s částmi kódu v jazyce symbolických adres) k selhání.

Volba \inikey{check_flags} specifikuje přepínače, které se mají nástroji
\shellcmd{cppcheck} předat při spuštění. Příkladem takového přepínače je
\texttt{--suppress=unusedFunction}, který potlačí všechna varování týkající se
funkcí, které jsou definovány, ale nejsou využity.
Nástroj není neomylný a v některých případech dojde k falešné detekci. Přepínač
\texttt{--inline-suppr} umožňuje potlačení varování pomocí speciálně
formátovaných komentářů přímo v analyzovaném zdrojovém kódu. Bližší informace
o formátu těchto speciálních komentářů lze nalézt v manuálu nástroje
\shellcmd{cppcheck} v sekci Inline suppressions.~\cite{cppcheckmanual}
Příklad jejich využití ale lze nalézt přímo ve firmware budíku.
V souboru \repopath{lib/SerialCLI/SerialCLI.h} je definováno pole
\lstinline[language=C]!char Serial_buffer_[kSerial_buffer_length_ + 1];!,
které je využíváno pro ukládání neúplného vstupu od uživatele. Vnitřní
implementace třídy \texttt{SerialCLI} nevyžaduje, aby bylo toto pole po celou
dobu životnosti objektu inicializované a obsahovalo validní data (například
nulový znak \lstinline[language=C]!'\0'! v prvním prvku pole).
\shellcmd{cppcheck} ale generuje generuje varování
\begin{lstlisting}[style=terminal,breakatwhitespace=true]
lib/SerialCLI/SerialCLI.h:55: [medium:warning] Member variable 'SerialCLI::Serial_buffer_' is not initialized in the constructor. [uninitMemberVar]
\end{lstlisting}
Řešením by mohlo být pole inicializovat (v hlavičkovém souboru či v
konstruktoru), varování lze ale potlačit i komentářem umístěným nad definici
konstruktoru:
\begin{lstlisting}[language=C++]
class SerialCLI
{
public:
    // ...

    // cppcheck-suppress uninitMemberVar symbolName=SerialCLI::Serial_buffer_
    SerialCLI(Stream& ser, const command_t* commands, const byte command_count,
              void (&print_error)(error_t),
              void (&cmd_not_found)(), const char* prompt
             ) : ser_(ser), commands_(commands), command_count_(command_count),
                 print_error_(print_error), cmd_not_found_(cmd_not_found),
                 prompt_(prompt) { }
\end{lstlisting}

% https://docs.platformio.org/en/latest/plus/pio-check.html

Díky nasazení statické analýzy kódu bylo ve firmware budíku odhaleno několik
potenciálních chyb (čtení z neinicializované proměnné apod.) a stylistických
chyb či nedostatků. Mnoho konstruktorů s jedním parametrem bylo deklarováno bez
klíčového slova \lstinline[language=C++]!explicit!. To mohlo vést ke vzniku
chyb kvůli nezamýšleným implicitním konverzím. Použití
\lstinline[language=C++]!explicit! je vyžadováno například standardem
MISRA C++.
% http://www.tlemp.com/download/rule/MISRA-CPP-2008-STANDARD.pdf ; ale asi to neni legalne ziskane
% 12-1-3

Příkladem detekované neinicializované proměnné je
\lstinline[language=C++]!PWMDimmer::value_!. Ta je definována v hlavičkovém
souboru \repopath{lib/PWMDimmer/PWMDimmer.h}. Konstruktor objektu
\texttt{PWMDimmer} umístěný v souboru \repopath{lib/PWMDimmer/PWMDimmer.cpp}
této proměnné ale nepřiřazoval žádnou hodnotu. Následující metoda proto mohla
při zavolání na nově vytvořeném objektu vracet nesmyslné hodnoty:
\begin{lstlisting}[language=C++]
class PWMDimmer
{
public:
    // ...

    //! Get current value
    byte get_value() const { return byte(value_); }

    // ...
};
\end{lstlisting}
Pro člověka není taková chyba snadno odhalitelná, mimo jiné i kvůli umístění
konstruktoru v jiném souboru než definice proměnné.
Řešení bylo jednoduché, proměnné je nyní přiřazována hodnota při definici:
\begin{lstlisting}[language=C++]
class PWMDimmer
{
protected:
    int value_ = 0;
    // ...
};
\end{lstlisting}
% git format-patch -1 5d8f9ba
