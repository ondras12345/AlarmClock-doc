\subsection{Arduino a PlatformIO}
% TODO kecy o Arduino

PlatformIO\footnote{\url{https://platformio.org/}} je multiplatformní, na
architektuře nezávislý nástroj s otevřeným zdrojovým kódem pro vývoj embedded
systémů. Podporuje mnoho SDK (včetně Arduino) a nezávisí na konkrétním
vývojovém prostředí (IDE). Oproti Arduino IDE nabízí možnost pracovat
s příkazovým řádkem namísto grafického rozhraní a díky tomu umožňuje
jednoduchou automatizaci různých procesů ve vývoji. Konfigurační soubor
\filename{platformio.ini} také umožňuje nastavit všechny parametry kompilace,
včetně přímého předávání přepínačů kompilátoru \shellcmd{gcc}. Je možné
vytvořit projekt, který je kompatibilní s Arduino IDE i s PlatformIO, ale to
pouze ztěžuje využívání pokročilých funkcí PlatformIO a nepřináší žádné výhody.

Nespornou výhodou PlatformIO je také správa závislostí -- knihoven, bez kterých
není možné firmware zkompilovat. Tím, že v konfiguračním souboru
\filename{platformio.ini} zaznamenává jejich přesné verze, odstraňuje potřebu
vytvářet tuto dokumentaci ručně. Znalost verzí knihoven, se kterými byl
firmware testován, je kritická pro budoucí kompilování. Může se totiž stát, že
v knihovně dojde v novější verzi ke změně API či zanesení chyby. Přístup
Arduino IDE (knihovny sdílené mezi projekty, notifikace doporučující
aktualizaci kdykoli je dostupná novější verze) není pro složitější projekty
závisející na mnoha knihovnách vhodný.

Ukázka obsahu konfiguračního souboru \filename{platformio.ini}:
\begin{lstlisting}
; PlatformIO Project Configuration File
;
;   Build options: build flags, source filter
;   Upload options: custom upload port, speed and extra flags
;   Library options: dependencies, extra library storages
;   Advanced options: extra scripting
;
; Please visit documentation for the other options and examples
; https://docs.platformio.org/page/projectconf.html

[platformio]
src_dir = src/AlarmClock

[env:uno]
platform = atmelavr
board = uno
framework = arduino
lib_deps = 
	adafruit/RTClib@^2.0.2
	thomasfredericks/Bounce2@^2.70.0
	paulstoffregen/Encoder@^1.4.1
	marcoschwartz/LiquidCrystal_I2C@^1.1.4
	paulstoffregen/TimerOne@^1.1.0
\end{lstlisting}


\subsubsection{Automatické testy}
PlatformIO podporuje testování jednotlivých jednotek zdrojového kódu (unit
testing), a to jak na cílové platformě (embedded MCU), tak i na počítači
použitém pro vývoj (native).
% TODO platformio testy
