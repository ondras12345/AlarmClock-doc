\nonuchapter{Závěr}
Cílem tohoto projektu bylo vytvořit spolehlivý konfigurovatelný budík.
Firmware zařízení prošel několikaměsíčním testováním, při kterém nebyl objeven
žádný významný nedostatek. Budík uživatele spolehlivě probouzel v~nastavený
čas. Velice se osvědčila funkce postupného rozsvěcení výkonové \acs{LED} před
začátkem akustického buzení, kterou běžné budíky nedisponují.

Potenciál pro další rozšiřování funkcionality firmware je omezen nedostatkem
místa v~programové paměti flash použitého mikrokontroléru. Bylo by ale možné
uplatnit mnohé optimalizace (zkrácení textů vypisovaných v~textovém
konfiguračním rozhraní, odstranění málo využívaných funkcí, přechod na
úspornější knihovny), kterými by se situace zlepšila.

Nastavování časů a~parametrů buzení lze provádět několika pohodlnými způsoby
(displej budíku, textové konfigurační rozhraní, webová stránka). Všechny tyto
metody nabízí rychlé a~srozumitelné uživatelské rozhraní.

Díky předchozím zkušenostem autora a~rozsáhlému testování jednotlivých
elektronických obvodů na nepájivém kontaktním poli byla plně funkční již první
revize desky plošných spojů. Drobným nedostatkem byl zbytkový svit výkonové LED
ve stavu zhasnuto, který byl způsobem průchodem velmi malého proudu (řádově
\SI{100}{\nano\ampere}) obvodem stmívače. Tento problém byl odstraněn přidáním
jednoho rezistoru. Výrobní dokumentace byla aktualizována, SMD rezistor byl
připájen na \acs{DPS} první revize bez nutnosti výroby nového kusu.

Blok zvukového výstupu splňuje vznesené požadavky. Pracuje bez nutnosti
samostatného napájení napětím jiným než \SI{5}{\volt}, ve stavu vypnutí
neprodukuje žádný slyšitelný zvuk a~umožňuje vytvářet tóny různých frekvencí
a~amplitud. Zjednodušením a~zobecněním implementace uživatelem definovatelných
zvuků buzení bylo dokonce umožněno přehrávání jednoduchých melodií. Praktický
přínos této funkce ale není příliš velký.

V~příští revizi hardware by bylo vhodné upravit obvod LC filtru tak, aby bylo
dosaženo nižšího harmonického zkreslení výstupního akustického signálu.
Zkreslení nepůsobí příliš rušivě a~pro účely vzbuzení uživatele není překážkou,
oproti dříve testovanému obvodu s~RC filtrem a~lineárním zesilovačem je ale
výstupní signál spínaného zesilovače budíku méně kvalitní.

DPS by bylo možné bez větších úprav uvést do sériové výroby. Bylo by ale nutné
provést revizi kusovníku, protože poměrně velká část použitých elektronických
součástek byla zakoupena od maloobchodního dodavatele GES Electronics, který
v~průběhu tohoto roku ukončil svou činnost. Velkým problémem by byl
i~celosvětový nedostatek čipů \texttt{ATmega328P}.

Navržená krabička vyrobená z~překližky je plně vyhovující pro tento prototyp.
Technologie výroby obráběním \acs{CNC} frézou umožnila přizpůsobit její tvar
a~rozměry požadavkům výrobku. Proces výroby ale zahrnuje mnoho zdlouhavých
manuálních kroků, především lepení. Pro výrobu většího počtu kusů by
pravděpodobně bylo nutné mechanický návrh změnit, nebo zvolit jednu z~komerčně
nabízených plastových krabiček.
