\section*{Úvod}  % TODO uvod se nema cislovat, ale pozor na obsah
Cílem tohoto projektu je vytvořit budík s pokročilými funkcemi, který lze
snadno vyrobit v amatérských podmínkách a přizpůsobit tak, aby vyhovoval
i požadavkům náročného uživatele--programátora. % TODO

Cílem není vytvořit zařízení přímo připojené k síti Internet. Vzhledem
k charakteru zařízení a skutečnosti, že bude umístěné poblíž hlavy uživatele
po několik hodin denně nepovažuje autor této práce za vhodné využití
bezdrátového přenosu informací. Alternativou by byla technologie Ethernet.
Výhody takového řešení oproti připojení k počítači rozhraním UART ale nejsou
příliš velké a nutnost obsluhovat síťové funkce ve firmware mikrokontroléru by
výrazně omezila možnosti implementace dalších funkcionalit (například zvukového
výstupu).

Dokumentace zdrojového kódu firmware je generována z komentářů pomocí
svobodného nástroje s otevřeným zdrojovým kódem \shellcmd{doxygen}. To
usnadňuje orientaci v projektu a tím umožňuje, aby si i běžný uživatel se
základní znalostí programování implementoval dodatečné funkce, které potřebuje.
% TODO footnote free software


