\nonuchapter{Úvod}
Cílem tohoto projektu je vytvořit budík s pokročilými funkcemi, který lze
snadno vyrobit v amatérských podmínkách a přizpůsobit tak, aby vyhovoval
i požadavkům náročného uživatele--programátora. % TODO

Cílem není vytvořit zařízení přímo připojené k síti Internet. Vzhledem
k charakteru zařízení a skutečnosti, že bude umístěné poblíž hlavy uživatele
po několik hodin denně nepovažuje autor této práce za vhodné využití
bezdrátového přenosu informací. Alternativou by byla technologie Ethernet.
Výhody takového řešení oproti připojení k počítači rozhraním UART ale nejsou
příliš velké a nutnost obsluhovat síťové funkce ve firmware mikrokontroléru by
výrazně omezila možnosti implementace dalších funkcionalit (například zvukového
výstupu).

Dokumentace zdrojového kódu firmware je generována z komentářů pomocí
svobodného nástroje s otevřeným zdrojovým kódem \shellcmd{doxygen}. To
usnadňuje orientaci v projektu a tím umožňuje, aby si i běžný uživatel se
základní znalostí programování implementoval dodatečné funkce, které potřebuje.

TODO vse free
software\footnote{\url{https://www.gnu.org/philosophy/free-sw.html}}.

Mnoho požadavků vychází ze zkušeností s komerčně dostupnými řešeními. Autor
této práce po několik let používal radiobudík Sencor SRC~330~GN.
% TODO obrazek sencor budik
Během používání bylo odhaleno několik problému, především nedostatek
nastavitelných časů buzení (pouze 2), nemožnost přiřadit budicímu času
libovolnou skupinu dní v týdnu a problémy se zrušením budíku v průběhu odložení
funkcí snooze. Nepříjemná byla také chyba ve firmware, která se
projevovala pískáním, které šlo zrušit pouze odpojením napájení budíku,
ke kterému docházelo náhodně v první hodině po stisku tlačítka nastavujícího
dobu nečinnosti, po které dojde ke zhasnutí displeje. Protože tento komerčně
dostupný budík neumožňuje zásah uživatele do firmware, neexistuje žádná
možnost, jak tyto nedostatky odstranit. Mezi další problémy patřilo
i slyšitelné bzučení o frekvenci \SI{50}{\hertz} vydávané interním
transformátorem (zařízení je napájené síťovým napětím).
