\subsection{Zvukový výstup}
Aby mohl budík plnit svou funkci, musí mít možnost vytvářet akustické signály.
To je možné řešit mnoha jednoduchými způsoby, například piezoelektrickým
bzučákem. Aby bylo akustické buzení pro uživatele přijatelnější, je vhodné
plynule zvyšovat amplitudu akustického signálu. Výhodná je také možnost měnit
frekvenci zvuku, aby mohl uživatel snadno rozlišit různé typy buzení podle
tónu.

Požadavky na blok zvukového výstupu jsou tedy
\begin{itemize}[nosep]
    \item nastavitelná frekvence (v rozsahu alespoň \SI{50}{\hertz} až % TODO
        \SI{8}{\kilo\hertz}),
    \item nastavitelná amplituda (s rozlišením 8 bitů, tedy ve 256 krocích),
    \item práce s nesymetrickým napájením \SI{5}{\volt},
    \item dokonalé ticho ve stavu softwarového vypnutí zvukového výstupu,
    \item vysoká účinnost,
    \item míra zkreslení dostatečné nízká na to, aby na uživatele nepůsobila
        příliš rušivě (obdélníkový signál není přijatelný).
\end{itemize}

Pro generování harmonického (sinusového) analogového signálu akustických
frekvencí (D/A převod) pomocí mikrokontroléru můžeme využít přímý převod pomocí
odporové sítě nebo nepřímý převod pomocí pulzně šířkové modulace (PWM).

Přímý DAC (D/A převodník) s dostatečným rozlišením by vyžadoval využití osmi
výstupních pinů, těch je ale u použitého MCU nedostatek. Oproti tomu PWM signál
můžeme snadno generovat pomocí vestavěného čítače/časovače a výstupem je
obdélníkový signál s časově proměnnou střídou na jednom výstupním pinu. Po
průchodu tohoto signálu dolní propustí získáváme analogový signál, jehož
okamžitou hodnotu určuje střída PWM signálu v daném okamžiku.

Aby bylo dosaženo dostatečného časového rozlišení (počet \uv{vzorků} za dobu
jedné periody výsledné sinusoidy), musí být frekvence PWM signálu
několikanásobně vyšší než frekvence sinusového signálu. Poměrně vysoká
frekvence PWM signálu navíc zjednodušuje návrh potřebné dolní propusti, protože
potřebná strmost její útlumové frekvenční charakteristiky není tak velká.

Příliš vysoká frekvence PWM signálu zase přináší problémy s implementací na
straně MCU. Jako optimální se jeví frekvence kolem \SI{50}{\kilo\hertz}.

\subsubsection{Zesilovač}
Z hlediska zesilování výstupního signálu máme dvě možnosti -- můžeme zesilovat
vyfiltrovaný sinusový signál pomocí lineární zesilovač (zesilovač pracující
v třídě A nebo AB), nebo můžeme zesilovat obdélníkový PWM signál \uv{spínaným}
zesilovačem (obdoba třídy D) a filtrovat až zesílený signál.


\paragraph{Lineární zesilovač}
Při prvotním vývoji zvukového výstupu byl z důvodu jednoduchosti využit
lineární zesilovač založený na integrovaném obvodu TDA2030. Pro přizpůsobení
signálu na úroveň vhodnou pro zpracování zesilovačem byl použit obvod zobrazený
na schématu na obrázku~\vref{fig:zvuk filtr linearni}.

\begin{figure}[htb]
    \centering
    \begin{circuitikz}
        \draw
            (0,2) to [R=$R_1$, a=$\SI{1}{\kilo\ohm}$, o-] (3,2)
            to [R=$R_2$, a=$\SI{10}{\kilo\ohm}$] (6,2)
            to [short] (9,2)
            to [short, -o] (10,2)
            (3,2) to [C, l_=$C_1$, a^=$\SI{10}{\nano\farad}$, *-] (3,0)
            to [short, -o] (0,0)
            (6,2) to [R, l_=$R_3$, a^=$\SI{1}{\kilo\ohm}$, *-] (6,0)
            to [short, -*] (3,0)
            (9,2) to [C, l_=$C_2$, a^=$\SI{10}{\nano\farad}$, *-] (9,0)
            to [short, -*] (6,0)
            (9,0) to [short, *-o] (10,0)
            ;
    \end{circuitikz}
    \caption{Schéma zapojení jednoduchého filtru a útlumového článku
        umožňujícího využití zesilovače s TDA2030 pro ověření funkce zvukového
        výstupu}
    \label{fig:zvuk filtr linearni}
\end{figure}

\begin{figure}[htb]
    \centering
    \input{sim/graf-zvuk-lin-filtr.tex}
    \caption{%
        Útlumová frekvenční charakteristika filtru zobrazeného na
        obrázku~\vref{fig:zvuk filtr linearni} podle simulace v LTspice
        % TODO zdrojak LTspice simulace do priloh
    }
    \label{fig:zvuk filtr linearni utlum}
\end{figure}

% TODO zapojeni zesilovace s TDA2030

Nevýhodou takového řešení je nízká účinnost zesilovače a poměrná složitost
obvodového řešení, kdy musíme signál vyfiltrovat, upravit na vhodnou amplitudu
a správně navrženým analogovým zesilovačem zesílit. Obvod také nesplňuje
požadavek na nesymetrické napájení napětím \SI{5}{\volt} (ale tento problém je
řešitelný použitím jiného integrovaného obvodu, například LM386).


\paragraph{Spínaný zesilovač}
Účinnějším (a z hlediska počtu součástek jednodušším) řešením je zesilování PWM
signálu spínacími tranzistory. Filtrace je řešena na výstupu zesilovače.

Protože máme k dispozici pouze nesymetrické napájení \SI{5}{\volt}, musíme
výstup spínacích prvků připojit na filtr a reproduktor přes sériový kondenzátor
odstraňující stejnosměrnou složku.
% TODO
