\subsection{PyAlarmClock}
Pro usnadnění manipulace s budíkem připojeným přes rozhraní UART z počítače
byla vytvořena knihovna \texttt{PyAlarmClock}. Je to modul jazyka Python
obsahující abstrakce pro práci s budíkem. Programátor vytvářející počítačový
program díky tomu nemusí ani znát konkrétní příkazy prostředí
\texttt{AlarmClockCLI}. Například nastavení data času modulu RTC se provádí
přiřazením objektu \texttt{datetime.datetime} vlastnosti \verb|RTC_time|
objektu \texttt{AlarmClock} bez nutnosti znát vnitřní implementaci, která
využívá příkazu \texttt{st} pro nastavení času a \texttt{sd} pro nastavení
data.

Pro ilustraci konceptu poslouží jeden z ukázkových programů~--
\repopath{examples/set_time.py}. V následující ukázce je navíc odstraněna část
zapínající podrobné protokolování všech událostí.
\begin{lstlisting}[language=Python,numbers=left]
#!/usr/bin/env python3

import PyAlarmClock
from datetime import datetime
import time

with PyAlarmClock.SerialAlarmClock('/dev/ttyUSB0') as ac:
    ac.RTC_time = datetime.now()
    time.sleep(1.65)  # RTC is polled every 0.8 seconds
    print(ac.RTC_time)
\end{lstlisting}
Program naváže spojení s budíkem, nastaví čas uložený v RTC na aktuální čas,
počká \SI{1,65}{\second} (aby stihl firmware budíku přečíst novou hodnotu
z RTC) a vypíše do konzole čas přečtený z budíku.
\begin{lstlisting}[basicstyle=\ttfamily\singlespacing]
$ ./test_dmp.py
2022-02-08 19:21:48
$
\end{lstlisting}


\subsubsection{MQTT adaptér}
Jedním z programů využívajících knihovnu \texttt{PyAlarmClock} je
\repopath{examples/mqtt_bridge.py}. Tento program umožňuje vzdálené ovládání
budíku zprávami přenášenými protokolem MQTT. Tímto způsobem lze zajistit
ovládání budíku několika programy současně bez nutnosti sdílet sériový port
mezi více procesy. Komunikace může probíhat nejen mezi více zařízeními ve
stejné počítačové síti, ale i v rámci jednoho serveru (\texttt{localhost}).

% TODO mqtt_bridge

% TODO availability / LWT
