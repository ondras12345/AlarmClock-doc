\subsection{Mikrokontrolér}
Volba MCU byla ovlivněna především autorovými předchozími zkušenostmi
s architekturou 8-bit AVR. Mikrokontroléry ATmega jsou navíc kompatibilní
s platformou Arduino\footnote{\url{https://www.arduino.cc/}}, která umožňuje
jednoduchý vývoj firmware v jazyce C++ bez nutnosti znát všechna specifika
daného MCU. To je v souladu s cíli projektu -- technicky zdatnější uživatel si
může snadno upravit firmware pro své potřeby.

Kromě rozšířenosti v amatérských projektech a dostupnosti vývojových desek ale
zvolený čip ATmega328P nevyniká počtem I/O pinů, výpočetním výkonem,
periferiemi ani cenou. Výhodou je ale jeho dostupnost nejen v pouzdrech pro
povrchovou montáž, ale i v pouzdře s vývody. Použití SMD součástky je totiž na
amatérsky vyrobeném plošném spoji složitější, obzvlášť pokud jde o plošný spoj
jednostranný.

\nocite{dshATmega328} % TODO info o MCU
