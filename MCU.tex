\section{Mikrokontrolér}
Volba MCU byla ovlivněna především autorovými předchozími zkušenostmi
s architekturou 8-bit AVR. Mikrokontroléry ATmega jsou navíc kompatibilní
s platformou Arduino\footnote{\url{https://www.arduino.cc/}}, která umožňuje
jednoduchý vývoj firmware v jazyce C++ bez nutnosti znát všechna specifika
daného MCU. To je v souladu s cíli projektu -- technicky zdatnější uživatel si
může snadno upravit firmware pro své potřeby.

Kromě rozšířenosti v amatérských projektech a dostupnosti vývojových desek ale
zvolený čip ATmega328P nevyniká počtem I/O pinů, výpočetním výkonem,
periferiemi ani cenou. Výhodou je ale jeho dostupnost nejen v pouzdrech pro
povrchovou montáž, ale i v pouzdře s vývody. Použití SMD součástky je totiž na
amatérsky vyrobeném plošném spoji složitější, obzvlášť pokud jde o plošný spoj
jednostranný.

\nocite{dshATmega328} % TODO info o MCU

I/O piny jsou děleny do třech portů (Port B, Port C, Port D). Podle toho jsou
piny číslovány~-- například \texttt{PB0}. Ve firmware jsou ale používána čísla
pinů frameworku Arduino. K vzájemnému převodu těchto dvou způsobů značení
slouží tabulka~\vref{tab:MCU pins}.

\begin{table}[htb]
    \centering
    \caption{%
        Číslování pinů MCU ATmega328P dle frameworku Arduino%
        %\footnote{\url{https://www.arduino.cc/en/uploads/Main/Arduino_Uno_Rev3-schematic.pdf}}
    }
    \label{tab:MCU pins}
    % https://www.abclinuxu.cz/tex/poradna/show/325037
    \catcode`\-=12
    \begin{tabular}{*{3}{ll}}
        \toprule
        \multicolumn{2}{c}{PORTD}
        & \multicolumn{2}{c}{PORTB}
        & \multicolumn{2}{c}{PORTC}
        \\
        \cmidrule(lr){1-2}
        \cmidrule(lr){3-4}
        \cmidrule(lr){5-6}
        Arduino pin     & MCU pin
        & Arduino pin     & MCU pin
        & Arduino pin     & MCU pin
        \\
        \midrule
        0   & PD0   & 8   & PB0   & A0  & PC0 \\
        1   & PD1   & 9   & PB1   & A1  & PC1 \\
        2   & PD2   & 10  & PB2   & A2  & PC2 \\
        3   & PD3   & 11  & PB3   & A3  & PC3 \\
        4   & PD4   & 12  & PB4   & A4  & PC4 \\
        5   & PD5   & 13  & PB5   & A5  & PC5 \\
        6   & PD6 \\
        7   & PD7 \\
        \bottomrule
    \end{tabular}
\end{table}
