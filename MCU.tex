\section{Mikrokontrolér}
Volba MCU byla ovlivněna především autorovými předchozími zkušenostmi
s architekturou 8-bit AVR. Mikrokontroléry ATmega jsou navíc kompatibilní
s platformou Arduino\footnote{\url{https://www.arduino.cc/}}, která umožňuje
jednoduchý vývoj firmware v jazyce C++ bez nutnosti znát všechna specifika
daného MCU. To je v souladu s cíli projektu -- technicky zdatnější uživatel si
může snadno upravit firmware pro své potřeby.

Kromě rozšířenosti v amatérských projektech a dostupnosti vývojových desek ale
zvolený čip ATmega328P nevyniká počtem I/O pinů, výpočetním výkonem,
periferiemi ani cenou. Výhodou je ale jeho dostupnost nejen v pouzdrech pro
povrchovou montáž, ale i v pouzdře s vývody. Použití SMD součástky je totiž na
amatérsky vyrobeném plošném spoji složitější, obzvlášť pokud jde o plošný spoj
jednostranný.

\nocite{dshATmega328} % TODO info o MCU

I/O piny jsou děleny do třech portů (Port B, Port C, Port D). Podle toho jsou
piny číslovány~-- například \MCUpin{PB0}. Ve firmware jsou ale používána čísla
pinů frameworku Arduino. K vzájemnému převodu těchto dvou způsobů značení
slouží tabulka~\vref{tab:MCU pins}.

\begin{table}[htb]
    \centering
    \caption{%
        Číslování pinů MCU ATmega328P dle frameworku Arduino%
        %\footnote{\url{https://www.arduino.cc/en/uploads/Main/Arduino_Uno_Rev3-schematic.pdf}}
    }
    \label{tab:MCU pins}
    % https://www.abclinuxu.cz/tex/poradna/show/325037
    \catcode`\-=12
    \begin{tabular}{*{3}{l>{\MCUpin\bgroup}l<{\egroup}}}
        \toprule
        \multicolumn{2}{c}{\texttt{PORTD}}
        & \multicolumn{2}{c}{\texttt{PORTB}}
        & \multicolumn{2}{c}{\texttt{PORTC}}
        \\
        \cmidrule(lr){1-2}
        \cmidrule(lr){3-4}
        \cmidrule(lr){5-6}
        Arduino pin     & \multicolumn{1}{c}{MCU pin}
        & Arduino pin     & \multicolumn{1}{c}{MCU pin}
        & Arduino pin     & \multicolumn{1}{c}{MCU pin}
        \\
        \midrule
        0   & PD0   & 8   & PB0   & A0  & PC0 \\
        1   & PD1   & 9   & PB1   & A1  & PC1 \\
        2   & PD2   & 10  & PB2   & A2  & PC2 \\
        3   & PD3   & 11  & PB3   & A3  & PC3 \\
        4   & PD4   & 12  & PB4   & A4  & PC4 \\
        5   & PD5   & 13  & PB5   & A5  & PC5 \\
        6   & PD6 \\
        7   & PD7 \\
        \bottomrule
    \end{tabular}
\end{table}

Mikrokontrolér je provozován s externím krystalovým rezonátorem. Pro zachování
kompatibility s platformou Arduino byla zvolena taktovací frekvence
\SI{16}{\mega\hertz}. V souladu s pokyny uvedenými v datasheetu MCU je krystal
připojen mezi piny \MCUpin{XTAL1} a \MCUpin{XTAL2}. Z každého z těchto pinů je
také přidán keramický kondenzátor o kapacitě \SI{22}{\pico\farad} do země.
Podle oficiálního schématu zapojení
\footnote{\url{https://www.arduino.cc/en/uploads/Main/Arduino_Uno_Rev3-schematic.pdf}}
Arduino UNO Rev3 je na těchto vývojových deskách osazován ještě rezistor
o hodnotě \SI{1}{\mega\ohm} paralelně ke krystalu (u hlavního MCU ATMEGA328-PU
je použit keramický rezonátor, stejný rezistor je ale i u krystalového
oscilátoru čipu ATMEGA16U2-MU). Ten ale není pro funkci obvodu potřebný
a datasheet MCU neuvádí, že by bylo nutné či vhodné ho
použít~\cite{dshATmega328}.
% https://www.avrfreaks.net/forum/atmega328p-1m-resistor-crystal
