\documentclass[
    utf8,
    aspectratio=169,
    %12pt,
    17pt,  % This is about the default size of PowerPoint and OpenOffice.org Impress.
]{beamer}
%\usetheme{Boadilla}
\setbeamertemplate{frametitle}{\insertframetitle\\\usebeamerfont{framesubtitle}\insertframesubtitle}
\setbeamerfont{frametitle}{size=\normalsize}
\setbeamerfont{framesubtitle}{size=\small}

\usepackage[czech]{babel}
\usepackage[T1]{fontenc}
\frenchspacing

\usepackage[binary-units]{siunitx}
\usepackage{listings}
%\usepackage[EFvoltages,siunitx,european, american inductors]{circuitikz}
\usepackage[
    %disable
    % todonotes can do weird stuff with circuitikz/tikz node[above], even when
    % disabled
]{todonotes}
\usepackage{lcd}
\usepackage{tikz}
\usetikzlibrary{fit}
\usetikzlibrary{positioning}
\usetikzlibrary{shapes.arrows}

\graphicspath{{figures/}{figures/prezentace/}}

\sisetup{
    output-decimal-marker = {,},
    list-separator = {; },
    list-final-separator = { a~},
    list-pair-separator = { a~},
    group-digits,
    group-minimum-digits=4,
    %range-phrase=\text{--},
    range-units=single,
    exponent-product=\ensuremath{\cdot}
}

%\DeclareSIUnit{\Kc}{K\check{c}}  % spatny znak
\DeclareSIUnit{\Kc}{\text{Kč}}
\DeclareSIUnit{\ks}{ks}

\makeatletter
\@ifpackageloaded{setspace}{%
    % https://tex.stackexchange.com/questions/55802/latex-lstinline
    \newcommand{\mysinglespacing}{%
      \setstretch{1}% no correction afterwards
    }
}{%
    \newcommand{\mysinglespacing}{}
}
\makeatother


\newcommand{\lstpostbreak}{\mbox{$\hookrightarrow$\space}}

% https://tex.stackexchange.com/questions/30512/how-to-insert-code-with-accents-with-listings
\lstset{
  basicstyle=\mysinglespacing\fontfamily{lmtt}\selectfont,
  columns=flexible,
  breaklines=true,
  postbreak=\lstpostbreak,
  showstringspaces=false,
  literate=%
    {á}{{\'a}}1
    {í}{{\'i}}1
    {é}{{\'e}}1
    {ý}{{\'y}}1
    {ú}{{\'u}}1
    {ó}{{\'o}}1
    {ě}{{\v{e}}}1
    {š}{{\v{s}}}1
    {č}{{\v{c}}}1
    {ř}{{\v{r}}}1
    {ž}{{\v{z}}}1
    {ď}{{\v{d}}}1
    {ť}{{\v{t}}}1
    {ň}{{\v{n}}}1
    {ů}{{\r{u}}}1
    {Á}{{\'A}}1
    {Í}{{\'I}}1
    {É}{{\'E}}1
    {Ý}{{\'Y}}1
    {Ú}{{\'U}}1
    {Ó}{{\'O}}1
    {Ě}{{\v{E}}}1
    {Š}{{\v{S}}}1
    {Č}{{\v{C}}}1
    {Ř}{{\v{R}}}1
    {Ž}{{\v{Z}}}1
    {Ď}{{\v{D}}}1
    {Ť}{{\v{T}}}1
    {Ň}{{\v{N}}}1
    {Ů}{{\r{U}}}1
}
  %literate={█}{{\#}}1

\lstdefinestyle{numbers}{
    numbers=left,
    xleftmargin=2em,
    %framexleftmargin=1.5em,  % cisla radku uvnitr ramecku, vypada hnusne
}

\lstdefinestyle{terminal}{
    basicstyle=\ttfamily\mysinglespacing,
    basewidth={0.5em,0.5em},
    frame=single,
    frameround=tttt,
}


\lstdefinelanguage{Ini}
{
    morecomment=[s][\bfseries]{[}{]},
    morecomment=[l]{\#},
    morecomment=[l]{;},
    morekeywords={},
    literate =%
        {-}{{-}}1
}

\lstdefinelanguage{yaml}
{
    columns=[l]flexible,
    %
    morekeywords={%
        true,false,null,
        !secret,!include,!include_dir_list,!include_dir_named,
        !include_dir_merge_list,!include_dir_merge_named
    },
    alsoletter={!},
    sensitive=false,
    %
    comment=[l]{\#},
    %
    morestring=[d]',
    morestring=[b]",
    %stringstyle=\mdseries\ttfamily\itshape,  % only for testing
    showstringspaces = false,
    % multiline strings >- -- looks too hard to implement
    %
    %moredelim=[l][\color{orange}]{\&},
    %moredelim=[l][\color{magenta}]{*},
    %
    % TODO YAML syntax highlighting - \bfseries keys not compatible with multiline strings
    %basicstyle=\bfseries\mysinglespacing,
    %moredelim=**[il][\mdseries{:}]{:}, % switch to value style at :
    %moredelim=[l][\mdseries]{:}, % switch to value style at :
    %
    % alsolanguage jinja2 -- too complicated
    %
    literate =  {---}{{\llap{\mdseries-{-}-}}}3
                {>}{{\textgreater}}1
                {>-}{{\textgreater-}}2
                {|}{{\textbar}}1
                {|-}{{\textbar-}}2
                % make dashes shorter
                {-}{{-}}1
                {-\ }{{\mdseries-\ }}2,
}

\lstdefinelanguage{Poke}
{
    sensitive=true,
    % see poke.vim in poke 2.1
    morekeywords={%
        % statement
        assert,break,continue,return,
        type,unit,fun,method,
        var,
        % operators
        in,sizeof,as,isa,unmap,
        % conditionals
        if,else,where,
        %
        struct,union,pinned,
        % loops
        while,for,
        % imports
        load,
        try,catch,until,raise,
        % exceptions - skipped
        % types
        string,void,int,uint,bit,nibble,
        byte,char,ushort,short,ulong,long,
        uint8,uint16,uint32,uint64,
        off64,uoff64,offset,
        Comparator,POSIX_Time32,POSIX_Time64,
        big,little,any,
        % constants
        ENDIAN_LITTLE,ENDIAN_BIG,
        IOS_F_READ,IOS_F_WRITE,IOS_F_CREATE,
        IOS_M_RDONLY,IOS_M_WRONLY,IOS_M_RDWR,
        load_path,NULL,OFFSET,
        % Poke std lib
        print,printf,catos,stoca,atoi,ltos,reverse,
        ltrim,rtrim,strchr,qsort,crc32,alignto,
        open,close,flush,get_ios,set_ios,iosize,
        rand,get_endian,set_endian,strace,exit,
        getenv,
        % integer literals - skipped
        % Units
        b,M,B,
        Kb,KB,Mb,MB,Gb,GB,
        Kib,KiB,Mib,MiB,Gib,GiB,
        % offsets - skipped
    },
    morecomment=[s]{/*}{*/},
    morecomment=[l]//,
    morestring=[b]",
    moredelim=[s][keywordstyle]{<}{>}  % uint<x>
}

% https://tex.stackexchange.com/questions/312/correctly-typesetting-a-tilde/160898#160898
\makeatletter
\newcommand\midtilde@raisedtilde[1][.5]{\raisebox{#1ex}{\texttildelow}}
\def\midtilde@normaltilde{\texttildelow}

\newcommand\midtilde%
{%
  \expandafter\in@\expandafter{\f@family}%
    {cmr,cmss,cmtt,cmm,cmsy,cmx,%
    lmr,lmss,lmtt,lmm,lmsy,lmx,%
    pxr,pxss,pxm,pxsy,pxx,%
    txr,txss,txm,txsy,txx}%
  \ifin@%
    \midtilde@raisedtilde%
  \else%
    \expandafter\in@\expandafter{\f@family}%
    {pxtt,txtt}%
    \ifin@%
      \midtilde@raisedtilde[.35]%
    \else%
      \midtilde@normaltilde%
    \fi%
  \fi%
}
\makeatother
\lstdefinelanguage{mybash}
{
    language = bash,
    literate =%
        % $# is not a comment...
        {\$\#}{{{\$\#}}}2 {\$\# }{{{\$\# }}}2
        % make dashes shorter
        {-}{{-}}1
        {~}{{\midtilde}}1
        % komentare s diakritikou
        {á}{{\'a}}1
        {í}{{\'i}}1
        {é}{{\'e}}1
        {ý}{{\'y}}1
        {ú}{{\'u}}1
        {ó}{{\'o}}1
        {ě}{{\v{e}}}1
        {š}{{\v{s}}}1
        {č}{{\v{c}}}1
        {ř}{{\v{r}}}1
        {ž}{{\v{z}}}1
        {ď}{{\v{d}}}1
        {ť}{{\v{t}}}1
        {ň}{{\v{n}}}1
        {ů}{{\r{u}}}1
        {Á}{{\'A}}1
        {Í}{{\'I}}1
        {É}{{\'E}}1
        {Ý}{{\'Y}}1
        {Ú}{{\'U}}1
        {Ó}{{\'O}}1
        {Ě}{{\v{E}}}1
        {Š}{{\v{S}}}1
        {Č}{{\v{C}}}1
        {Ř}{{\v{R}}}1
        {Ž}{{\v{Z}}}1
        {Ď}{{\v{D}}}1
        {Ť}{{\v{T}}}1
        {Ň}{{\v{N}}}1
        {Ů}{{\r{U}}}1
}

\lstdefinelanguage{Dockerfile}{
    keywords={FROM, RUN, COPY, ADD, ENTRYPOINT, CMD,  ENV, ARG, WORKDIR, EXPOSE, LABEL, USER, VOLUME, STOPSIGNAL, ONBUILD, MAINTAINER},
    sensitive=false,
    comment=[l]{\#},
    morestring=[b]',
    morestring=[b]",
    literate =%
        {-}{{-}}1
}

\lstdefinelanguage{myC++}{
    language=C++,
    morekeywords={%
        uint8_t, uint16_t, uint32_t,
        int8_t, int16_t, int32_t,
        size_t,
        constexpr,
        byte
    }
}

\lstdefinelanguage{hashcomment}{
    comment=[l]{\#},
}


% This file was automatically generated by ./script/LCDchars from prilohy/AlarmClock/lib/GUI/LCDchars.cpp.

\DefineLCDchar{home}{%
    00100%
    01110%
    11111%
    10001%
    10001%
    10001%
    11111%
    00000%
}

\DefineLCDchar{bell}{%
    00100%
    01110%
    01110%
    01110%
    01110%
    11111%
    00000%
    00100%
}

\DefineLCDchar{timer}{%
    11111%
    01110%
    00100%
    00100%
    01110%
    11111%
    00000%
    00000%
}

\DefineLCDchar{apply}{%
    00000%
    00000%
    00001%
    00010%
    10100%
    01000%
    00000%
    00000%
}

\DefineLCDchar{cancel}{%
    00000%
    10001%
    01010%
    00100%
    01010%
    10001%
    00000%
    00000%
}


% HD44780 neumi plnou diakritiku, neco ma v ROM A02, ale pak neumi podtrhavat
\DefineLCDchar{idlouhe}{%
    00010%
    00100%
    00000%
    01100%
    00100%
    %00100%
    00100%
    01110%
}
\DefineLCDchar{chacek}{%
    01010%
    00100%
    01110%
    10000%
    10000%
    10001%
    01110%
}

\newcommand{\fullsizegraphics}[2][]{%
    \centering%
    \includegraphics[width=\textwidth,height=0.70\textheight,keepaspectratio,#1]{#2}%
}
\def\TikZ{Ti\emph{k}Z}
\def\Circuitikz{Circui\TikZ}

\title{Chytrý budík}
\author{Ondřej Sluka}
%\institute{VOŠ a SPŠE Plzeň} % TODO
\date{2022}

\begin{document}
\frame{\titlepage}

\begin{frame}
    \frametitle{Motivace}
    \begin{columns}
        \begin{column}{0.5\textwidth}
            \begin{itemize}
                \item Špatné zkušenosti s~komerčně dostupnými budíky.
                \item Nelze upravit firmware
                \item Neopravitelné chyby
                \item Chybějící funkce
            \end{itemize}
        \end{column}
        \begin{column}{0.5\textwidth}
            \fullsizegraphics{sencor-SRC-330-GN}
        \end{column}
    \end{columns}
\end{frame}

\begin{frame}[fragile]
    \frametitle{Cíle}
    \begin{columns}
        \begin{column}{0.6\textwidth}
            \begin{itemize}
                \item \alert{Spolehlivost}
                \item Konfigurovatelnost
                %\item Jednoduchost výroby v amatérských podmínkách
                \item Jednoduchost úprav% -- dokumentace firmware
                \item Plná funkčnost bez připojení k~serveru
            \end{itemize}
        \end{column}
        \begin{column}{0.4\textwidth}
            \LCD{2}{16}!{home}0/5 12---67 OFF!
                       !12:34+01*1  25lS!
        \end{column}
    \end{columns}
\end{frame}

\begin{frame}
    \frametitle{Struktura projektu}
    \begin{itemize}
        \item Firmware
            \begin{itemize}
                \item Základní funkce
                \item Soběstačná jednotka
            \end{itemize}
        \item Hardware
            \begin{itemize}
                \item Požadavky určeny firmware
                \item Dlouhé experimentování na nepájivém kontaktním poli
            \end{itemize}
        \item Serverový software
            \begin{itemize}
                \item Doplňkové funkce
            \end{itemize}
    \end{itemize}
\end{frame}

\begin{frame}
    \frametitle{Deska plošných spojů}
    \begin{columns}
        \begin{column}{0.5\textwidth}
            \fullsizegraphics{PCB-top}
        \end{column}
        \begin{column}{0.5\textwidth}
            \fullsizegraphics{PCB-bot}
        \end{column}
    \end{columns}

\end{frame}

\begin{frame}
    \frametitle{Otázky od vedoucího práce}
    \begin{enumerate}
        \item Která hardwarová část práce byla nejobtížnější a~proč?
        \item Která softwarová část práce byla nejobtížnější a~proč?
    \end{enumerate}
\end{frame}

\begin{frame}[fragile]  % fragile because of verb
    \frametitle{Která hardwarová část práce byla nejobtížnější a~proč?}
    \framesubtitle{Výroba krabičky}
    \begin{columns}
        \begin{column}{0.5\textwidth}
            \fullsizegraphics[bb=0px 430px 2200px 1918px, clip]{final-front}
        \end{column}
        \begin{column}{0.5\textwidth}
            \begin{itemize}
                \item Návrh ve FreeCAD
                \item Definice procesu obrábění
                %\item Správný postprocesor G-kódu (Grbl nepodporuje
                %    \verb|G81|, \verb|G82|, \ldots)
                \item Obrábění překližky na malé CNC fréze
            \end{itemize}
        \end{column}
    \end{columns}
\end{frame}

\begin{frame}
    \frametitle{Která hardwarová část práce byla nejobtížnější a~proč?}
    \framesubtitle{Výroba krabičky}
    \only<1>{\fullsizegraphics{krabicka-CNC-repro}}
    \only<2>{\fullsizegraphics{krabicka-CNC-laser}}
    \only<3>{\fullsizegraphics{krabicka-CNC-poskozeni}}
\end{frame}


\begin{frame}
    \frametitle{Která hardwarová část práce byla nejobtížnější a proč?}
    \framesubtitle{Zvukový výstup}
    \only<1>{
    \begin{itemize}
        \item Nejdříve snaha generovat na MCU obdélníkový průběh
            a filtrovat v hardware
        \item Experimentování s existujícími knihovnami
        \item Problémy s pamětí $\Rightarrow$ vlastní implementace
        \item Lineární zesilovač
        \item \uv{Spínaný} zesilovač
    \end{itemize}
    }
    \only<2>{\fullsizegraphics{zvuk-stul1}}
    \only<3>{\fullsizegraphics{zvuk-stul2}}
\end{frame}


\begin{frame}[fragile]
    \frametitle{Která softwarová část práce byla nejobtížnější a proč?}
    \framesubtitle{Firmware}
    \begin{itemize}
        \item Vývoj trval nejdéle
        \item Největší nároky na spolehlivost
        \item Největší míra manuálního testování
            %(před vznikem \texttt{PyAlarmClock})
        %\item Testování v produkčním prostředí by bylo riskantní
        %    $\rightarrow$ více instancí
        %\item Omezené HW prostředky (\SI{32}{\kilo\byte} Flash,
        %    \SI{2}{\kilo\byte} SRAM)
    \end{itemize}
    {
        \footnotesize
        \begin{lstlisting}[style=terminal]
RAM:   [======    ]  62.9% (used 1289 bytes from 2048 bytes)
Flash: [==========]  99.8% (used 32182 bytes from 32256 bytes)
        \end{lstlisting}
    }
\end{frame}


\begin{frame}
    \frametitle{Která softwarová část práce byla nejobtížnější a proč?}
    \framesubtitle{Návrh celého systému}
    \begin{columns}
        \begin{column}{0.5\textwidth}
            \scriptsize
            \tikzstyle{box}=[draw, rounded corners=2mm, font={\bfseries}, align=center, inner sep=2mm]
            \begin{tikzpicture}[>=latex,scale=0.7]
                \node [align=center, font={\bfseries}] (mqtt_adapter label) at (0,0) {MQTT\\adaptér};
                \node [box, below=2mm of mqtt_adapter label] (PyAlarmClock) {PyAlarmClock};
                \node [box, fit={(mqtt_adapter label) (PyAlarmClock)}] (mqtt_adapter) {};
                \node [box, below=15mm of PyAlarmClock] (AlarmClock) {Budík\\ATmega328P};
                \draw [<->] (PyAlarmClock) -- (AlarmClock)
                    node [midway,sloped,above] {\tiny UART}
                    node [midway,sloped,below] {\tiny YAML}
                    ;
                \node [box, above=15mm of mqtt_adapter] (Home Assistant) {Home Assistant};
                \draw [<->] (Home Assistant) -- (mqtt_adapter)
                    node [midway,sloped,above] {\tiny MQTT}
                    node [midway,sloped,below] {\tiny JSON}
                    ;
                \node [box, left=5mm of Home Assistant] (AlarmClockWeb) {AlarmClockWeb};
                \draw [<->] (mqtt_adapter.north) -- (AlarmClockWeb)
                    node [midway,sloped,above] {\tiny MQTT}
                    node [midway,sloped,below] {\tiny JSON}
                    ;
                \node [align=center, font={\bfseries}] (PyAlarmClock tests label) at (AlarmClockWeb |- mqtt_adapter label) {Automatické\\testy};
                \node [box, below=2mm of PyAlarmClock tests label] (PyAlarmClock tests lib) {PyAlarmClock};
                \node [box, fit={(PyAlarmClock tests label) (PyAlarmClock tests lib)}] (PyAlarmClock tests) {};
                \draw [<->, dashed] (PyAlarmClock tests lib) -- (AlarmClock)
                    node [midway,sloped,above] {\tiny UART}
                    node [midway,sloped,below] {\tiny YAML}
                    ;
                \node [box, below=10mm of PyAlarmClock tests] (terminal) {Terminál};
                \draw [<->, dashed] (terminal) -- (AlarmClock)
                    node [midway,sloped,above] {\tiny UART}
                    node [midway,sloped,below] {\tiny text}
                    ;
            \end{tikzpicture}
        \end{column}
        \begin{column}{0.5\textwidth}
            \begin{itemize}
                \item Kolik částí, jaké úkoly?
                \item Jak sdílet jeden sériový port mezi více programy?
                \item Jak budou části komunikovat?
                    \begin{itemize}
                        \item HTTP API?
                        \item MQTT API?
                    \end{itemize}
            \end{itemize}
        \end{column}
    \end{columns}
\end{frame}


\begin{frame}
    \frametitle{Otázky od oponenta}
    \begin{enumerate}
        \item Jaké jsou použity melodie pro buzení?
        \item Jaký je finanční rozpočet budíku?
    \end{enumerate}
\end{frame}

\begin{frame}
    \frametitle{Jaké jsou použity melodie pro buzení?}
    \only<1>{
        \begin{itemize}
            \item Běžné buzení -- přerušovaný \num{1}\si{\kilo\hertz} tón
                (\SI{500}{\milli\second} tón, \SI{500}{\milli\second} ticho)
                s~postupně zvyšovanou amplitudou
            \item Neodkladné buzení -- \SI{2}{\kilo\hertz};
                \SI{250}{\milli\second}
            \item Složitější melodie nejsou primárním účelem
            %\item Chybí nástroj pro snadné vytváření (používám Python script -- začátek E1M1)
        \end{itemize}
    }
    \only<2>{
        \fullsizegraphics{zvuk-sine-1k-ramp-silence}
        % prvni rampa uz je mirnejsi, ale pri tehle casove zakladne to stejne
        % neni videt
    }
\end{frame}

\begin{frame}
    \frametitle{Jaký je finanční rozpočet budíku?}
    \begin{itemize}
        %\item Maloobchodní ceny
        \item Cena za 1 kus, ale u většiny minimum např. \SI{100}{\ks}
        \item GES vypnul e-shop $\rightarrow$ katalog 2010
        \item Např. MCU ATmega328P koupen za \SI{98}{\Kc}, ale dnes stojí
            \SI{230}{\Kc}
        \vspace{5mm}
        %\item Elektronické součástky \SI{510}{\Kc},
        %    cuprextitová destička \SI{28}{\Kc},
        %    reproduktor \SI{39}{\Kc}
        \item Elektronika \SI{580}{\Kc}
        \item Krabička: buková překližka
            \SI[product-units=single]{4x600x1200}{\milli\meter},
            \SI{259}{\Kc}, využito \SI{17}{\percent} $\Rightarrow$ \SI{44}{\Kc}
    \end{itemize}
\end{frame}



\begin{frame}
    \frametitle{Dokumentace}
    \begin{itemize}
        %\item Poměrně veliká část času
        \item {\rmfamily\LaTeX{}}, {\rmfamily\TikZ{}},
            {\rmfamily\Circuitikz{}}, Listings \ldots
        \item Vim, \texttt{git}, \texttt{make}, \ldots
            $\rightarrow$ v podstatě se neliší od vývoje software
        \item \texttt{git} submodules (firmware, PyAlarmClock, \ldots)
        \item Snaha automaticky generovat z ostatních repozitářů
    \end{itemize}
\end{frame}

\begin{frame}[fragile] % [fragile] is also needed for \LCD
    \frametitle{Dokumentace}
    %\framesubtitle{Sazba obrázků LCD displejů}
    \begin{center}
        \hfill
        \LCD{2}{16}!  LCD bal{idlouhe}{chacek}ek   !
                   !   pro LaTeX    !
        \hfill
        {
            \huge
            \textLCD{1}!{home} !
        }
        \hfill~
    \end{center}

    \begin{columns}
        \begin{column}{0.4\textwidth}
            \footnotesize
            Definice znaku ve firmware:
            \begin{lstlisting}[language=C++]
byte LCD_char_home[] = {
    B00100,
    B01110,
    B11111,
    B10001,
    B10001,
    B10001,
    B11111,
    B00000
};
            \end{lstlisting}
        \end{column}
        \begin{column}{0.4\textwidth}
            \footnotesize
            Definice znaku pro balíček \textLCD{3}|LCD|:
            \begin{lstlisting}[language={[LaTeX]TeX}]
\DefineLCDchar{home}{%
    00100%
    01110%
    11111%
    10001%
    10001%
    10001%
    11111%
    00000%
}
            \end{lstlisting}
        \end{column}
    \end{columns}
    \begin{tikzpicture}[overlay, remember picture]
        \node[single arrow, draw=black, fill=green,
            minimum width = 12pt, single arrow head extend=4pt,
            minimum height=12mm, % length of arrow
            below=0.1\textheight of current page.center,
            ] {\small\texttt{sed}};
    \end{tikzpicture}
    %\lstinputlisting[language=mybash]{scripts/LCDchars}
\end{frame}

\begin{frame}[fragile]
    \frametitle{Závěr}
    \framesubtitle{Budoucnost projektu}
    \begin{itemize}
        \item Revize BOM
        \item Více nastavení bez nutnosti rekompilace
        \item \verb|binary_sensor.asleep|
        \item (Oboustranná DPS)
        \item (EMC testování)
    \end{itemize}
\end{frame}

\end{document}
