\section{Indikace výpadku napájení}
Protože je budík napájen stejnosměrným napětím \SI{5}{\volt} získaným ze zdroje
závislého na síti, může dojít k nečekanému výpadku. V takovém případě musí
budík selhat bezpečně. Bezpečným selháním je okamžité vzbuzení uživatele, aby
nemohlo při dlouho trvajícím výpadku dojít k jeho zaspání. Tento přístup byl
zvolen protože zajištění plné funkce při výpadku napájení by vyžadovalo přidání
akumulátoru a složitých nabíjecích obvodů.

\begin{figure}
    \centering
    \includegraphics[page=1, clip, bb=90mm 12mm 176mm 58mm, width=0.80\textwidth]{prilohy/AlarmClock-hardware/AlarmClock/AlarmClock.pdf}
    \par\bigskip
    {\footnotesize Označení součástek neodpovídá simulaci na
    obrázku~\vref{fig:UPS RC sim sch}}
    \caption{Schéma zapojení obvodu akustické indikace výpadku napájení}
    \label{fig:UPS sch}
\end{figure}

Pro akustickou indikaci je využit aktivní bzučák, tedy piezoelektrický měnič
s interním generátorem budicího signálu. Pro jeho provoz je nutné pouze
stejnosměrné napájecí napětí (jmenovitá hodnota je \SI{3,3}{\volt}).
Jako zdroj energie pro bzučák je použita knoflíková baterie typu CR2032, která
slouží i jako záložní baterie hodin reálného času. Protože jde o primární
lithiový článek, je nutné zajistit minimální proudový odběr v obou klidových
stavech (připojené napájení s plnou funkčností budíku a odpojené napájení po
zaznění zvukového signálu). V době výpadku již neběží na MCU program, proto je
nutné implementovat funkčnost akustické indikace v hardware.

Obvod je navržen tak, že pokud je připojené napájení \SI{5}{\volt} a na vstupu
\texttt{buzzer} je napětí \SI{5}{\volt}, bzučák nepíská. Pokud je připojeno
napájení ale vstup \texttt{buzzer} je připojen k zemi či zcela odpojen, píská
bzučák nepřetržitě. Pokud dojde k výpadku napájení, neprotéká diodou ze vstupu
\texttt{buzzer} žádný proud a bzučák píská, dokud nedojde k vybití
elektrolytického kondenzátoru a zavření NPN tranzistoru (doba pískání je
přibližně \SI{3}{\second}). Rezistor s odporem \SI{100}{\kilo\ohm} připojený
paralelně ke kondenzátoru zajišťuje rychlejší vybíjení v okolí prahového napětí
přechodu báze--emitor NPN tranzistoru, který se díky tomu rychleji zavře.

Protože je v obvodu zahrnuta nelineární součástka (přechod báze--emitor NPN
tranzistoru), nelze napětí na vybíjejícím se kondenzátoru matematicky modelovat
jednoduchou klesající exponenciálou. Lze ale využít simulátor elektronických
obvodů (viz obrázek~\vref{fig:UPS RC sim}). Výpadek napájení je simulován
pomocí direktivy \verb|.ic|, která určuje počáteční podmínky simulace. Napětí
na kondenzátoru je v čase $t=0$ nastaveno na \SI{4,4}{\volt} (\SI{5}{\volt}
mínus úbytek napětí na křemíkové diodě).
% - vybijeni kondiku z (5V-0.6V) na 0.6V zabere 2tau
%   gnuplot> plot [0:5] 4.4*(exp(-x)), 0.6

\begin{figure}
    \centering
    \subcaptionbox{%
        Schéma zapojení a nastavení simulace%
        \label{fig:UPS RC sim sch}
    }{%
        \includegraphics[width=0.7\textwidth]{sim/cropped_vypadek-RC}
    }
    \subcaptionbox{%
        Časový průběh proudu protékajícího rezistorem $R_3$ po výpadku
        napájení%
        \label{fig:UPS RC sim proud}
    }{%
        \input{sim/graf-vypadek-RC.tex}
    }
    \caption{Simulace konečné verze lineárního stmívače v programu LTspice}
    \label{fig:UPS RC sim}
\end{figure}

Připojením vstupu \texttt{buzzer} na výstupní pin MCU můžeme zajistit, že
bzučák reaguje například i na vyjmutí MCU z patice či běh nesprávného programu,
který na tento výstup nenastavuje logickou 1.

Praktickým měřením bylo ověřeno, že proud protékající diodou v závěrném směru
(D1 na obrázku~\vref{fig:UPS sch}) nestačí ani po zesílení PNP tranzistorem Q2
pro znatelné zvýšení rychlosti vybíjení baterie BT1.
