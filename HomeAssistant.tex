\section{Home Assistant}
Home Assistant je svobodný software (licence Apache) pro domácí automatizaci se
zaměřením na soukromí. Uživatel jej provozuje na vlastním serveru (například
NAS či Raspberry Pi). Cílem projektu je \emph{automatizace}, ne vzdálené
ovládání.\footnote{\url{https://www.home-assistant.io/blog/2016/01/19/perfect-home-automation/}}.
K němu lze využít také, v obvyklých situacích je ale hodnota získaná možností
ovládat různá zařízení z mobilního telefonu přinejmenším sporná.

\subsection{Instalace}
Software Home Assistant je napsaný v jazyce Python je existuje tedy několik
způsobů, jak jen nainstalovat. Nejjednodušší a pravděpodobně nejrobustnější
možností je instalace v kontejneru nástroje Docker. Podrobnou dokumentaci
s návodem pro instalaci lze nalézt na webových stránkách
\url{https://www.home-assistant.io/installation/}.
V prostředí operačního systému Debian 10 (buster) na Raspberry Pi 4 byla
instalace provedena následujícím způsobem:
\begin{lstlisting}[language=mybash]
# Stáhneme instalační skript dockeru.
cd ~/temp/ || mkdir ~/temp && cd ~/temp || exit 1
curl -fsSL https://get.docker.com -o get-docker.sh
# Jde o spustitelný soubor stažený z Internetu,
# je tedy nutné zkontrolovat jeho obsah a přesvědčit se,
# že není škodlivý.
less get-docker.sh
# Provedeme instalaci jako uživatel root.
sudo sh get-docker.sh

# Vytvoříme uživatele homeassistant a přidáme mu práva pro práci s dockerem.
sudo useradd -rm homeassistant -G docker
# Vytvoříme funkci umožňující snadné spouštění příkazů pod tímto uživatelem.
hau(){
    sudo -u homeassistant "$@"
}

hau mkdir /home/homeassistant/hass-config

# Stáhneme a spustíme vlastní kontejner.
hau docker run -d \
    --name homeassistant \
    --restart=unless-stopped \
    -e TZ=Europe/Prague \
    -v /home/homeassistant/hass-config:/config \
    --network=host \
    --privileged \
    "ghcr.io/home-assistant/home-assistant:stable"
\end{lstlisting}

Nová verze Home Assistant je vydávána každý měsíc (například verze 2022.3.0
v březnu roku 2022), v průběhu měsíce jsou navíc vydávány opravné verze
(2022.3.1; 2022.3.2 atd.). V době psaní tohoto textu je nejnovější stabilní
verze 2022.3.2.
Proces aktualizace je poměrně jednoduchý, obsahuje ale několik kroků:
\begin{enumerate}
    \item (volitelně) odstranění starých verzí kontejneru pro úsporu místa na
        systémovém disku,

    \item stažení nového image kontejneru příkazem \shellcmd{docker pull},

    \item zastavení starého kontejneru (\shellcmd{docker stop homeassistant}),

    \item odstranění starého kontejneru (\shellcmd{docker rm homeassistant}),

    \item (volitelně) záloha konfiguračních souborů a databáze,

    \item spuštění nového kontejneru příkazem \shellcmd{docker run} se stejnými
        přepínači jako při instalaci.
\end{enumerate}
U procesů s vícero netriviálními kroky strmě narůstá riziko lidské chyby,
proto byl vytvořen jednoduchý skript, který postup aktualizace automatizuje.
Jeho zdrojový kód je otištěn v příloze~\vref{app:update ha}.

Po spuštění poslouchá webový server Home Assistant na portu \port{8123/tcp}.
Pokud je na serveru provozován firewall, je nutné tento port otevřít pro
zařízení, která mají mít k webovému rozhraní přístup. S jednoduchým nástrojem
pro konfiguraci \shellcmd{iptables} nazvaného \shellcmd{ufw}
(\foreignlanguage{english}{Uncomplicated FireWall} to lze provést následovně:
\begin{lstlisting}[language=mybash]
sudo tee /etc/ufw/applications.d/homeassistant << EOF
[homeassistant]
title=Home Assistant Web UI
description=Home Assistant Web UI
ports=8123/tcp
EOF
sudo ufw app update homeassistant

# Výpis informací o aplikaci:
sudo ufw app info homeassistant

# Povolení komunikace pro konkrétní zařízení:
sudo ufw allow from 192.168.1.4 to any app homeassistant comment "Home Assistant moje PC"
\end{lstlisting}


\subsection{Konfigurace}
Konfigurace Home Assistant se provádí zčásti přímo z webového rozhraní a zčásti
pomocí konfiguračních souborů ve formátu YAML. Projekt budíku komunikuje s Home
Assistant pomocí protokolu MQTT, zaměříme se proto pouze na konfiguraci pomocí
YAML souborů.

Veškerou konfiguraci by bylo možné vkládat do hlavního konfiguračního souboru
\filename{configuration.yaml}.
Z hlediska zachování přehlednosti a jednoduchosti údržby je ale vhodnější
konfiguraci rozdělit do více souborů.
Můžeme využít funkci
\texttt{packages}\footnote{\url{https://www.home-assistant.io/docs/configuration/packages/}},
která nám umožní členit konfiguraci podle libovolného kritéria (například podle
fyzického zařízení). Běžná metoda využívající
\lstinline[language=yaml]|!include nazev_souboru.yaml| totiž umožňuje pouze
členění podle domény (\HAdomain{sensor}, \HAdomain{binary_sensor},
\HAdomain{light}, ...). Takto vytvoření balíček konfigurace je také vhodný pro
následnou distribuci, protože jeho použití je naprosto triviální.

V hlavním konfiguračním souboru \filename{configuration.yaml} přidáme záznam,
který určuje, že jednotlivé balíčky konfigurace se nacházejí v adresáři
\filename{packages/}.
\begin{lstlisting}[language=yaml]
homeassistant:
  packages: !include_dir_named packages
\end{lstlisting}
V tomto adresáři poté můžeme vytvořit soubor
\filename{packages/alarmclock.yaml}, který obsahuje úplnou konfiguraci všech
entit tvořících budík.
\begin{lstlisting}[language=yaml]
switch:
  - platform: mqtt
    name: AlarmClock inhibit
    state_topic: alarmclock/stat/inhibit
    command_topic: alarmclock/cmnd/inhibit
    availability_topic: alarmclock/stat/available


light:
  - platform: mqtt
    name: AlarmClock lamp
    state_topic: alarmclock/stat/lamp
    command_topic: alarmclock/cmnd/lamp
    availability_topic: alarmclock/stat/available

# ...
\end{lstlisting}
\todo[inline]{repozitar s HA konfiguraci}


\todo[inline]{obecne info o HA}
% https://www.home-assistant.io/docs/glossary/
% jinja2 templates

\subsection{Integrace s budíkem}
MQTT adaptér \shellcmd{ac2mqtt} popisovaný v předchozím textu
umožňuje snadnou integraci s tímto systémem. Platforma \texttt{mqtt} nabízí
v prostředí Home Assistant možnost vytvářet entity z celé řady domén~--
například \HAdomain{sensor}, \HAdomain{binary_sensor}, \HAdomain{light},
\HAdomain{switch} a další.

Příkladem funkce budíku, kterou lze do tohoto systému snadno integrovat, je
ovládání světla \uv{lamp}. To je určeno primárně pro buzení, není ale důvod,
proč by nemohlo být využíváno i pro jiné účely. Protože jde o osvětlení,
využijeme typ entity
\href{https://www.home-assistant.io/integrations/light.mqtt}{MQTT Light}.
Vlastní konfigurace je triviální:
\begin{lstlisting}[language=yaml]
light:
  - platform: mqtt
    name: AlarmClock lamp
    state_topic: alarmclock/stat/lamp
    command_topic: alarmclock/cmnd/lamp
    availability_topic: alarmclock/stat/available
\end{lstlisting}
Využíváme totiž výchozích hodnot -- například \lstinline!payload_off: OFF!
a~\lstinline!payload_on: ON!. Ty se shodují se zprávami odesílanými programem
\shellcmd{ac2mqtt}. \yamlkey{availability_topic} složí k detekci
nefunkčního spojení s budíkem. V takovém případě přejde entita
\HAentity{light.alarmclock_lamp} do stavu \HAstate{unavailable}.

Platforma \foreignlanguage{english}{MQTT Light} umožňuje i ovládání jasu
světla, čehož využíváme u stmívatelného LED pásku \uv{ambient}. Konfigurace je
o něco složitější.
\begin{lstlisting}[language=yaml]
light:
  - platform: mqtt
    schema: template
    name: AlarmClock ambient
    availability_topic: alarmclock/stat/available
    command_topic: alarmclock/cmnd/ambient
    state_topic: alarmclock/stat/ambient
    state_template: "{{ 'off' if value == '0' else 'on' }}"
    brightness_template: "{{ value }}"
    command_off_template: 0
    command_on_template: >
      
      {{ brightness }}
      
      255
      
\end{lstlisting}
Protože je LED pásek řízen pouze jasem, kde hodnota \num{255} představuje plný
jas a \num{0} představuje zhasnuté světlo, je stav entity (zapnuto / vypnuto)
z tohoto čísla. To je úlohou šablony \yamlkey{state_template}. Protože proměnná
\texttt{value} představuje textový řetězec přijatý na \yamlkey{state_topic},
porovnáváme ji s textovým řetězcem \lstinline[language=Python]!'0'!. Druhou
možností by bylo převést \texttt{value} na celočíselný datový typ filtrem
\texttt{int} a porovnávat čísla.

\yamlkey{brightness_template} určuje, jak z přijaté MQTT zprávy odvodit jas
světla jako číslo v rozsahu \numrange{0}{255}.

\yamlkey{command_off_template} určuje zprávu, která se má poslat na
\yamlkey{command_topic}, když je požadováno vypnutí světla. Protože vypnutí
provádíme nastavením jasu na \num{0}, není ani potřeba využít šablonu.

\yamlkey{command_on_template} je složitější, protože musí správně reagovat
v různých situacích:
\begin{enumerate}[nosep]
    \item Je požadováno zapnutí světla, ale není specifikován jas. V takovém
        případě není proměnná \texttt{brightness} definována a do
        \yamlkey{command_topic} se pošle hodnota \num{255}.
    \item Je požadované zapnutí světla s určitou hodnotou jasu. V takovém
        případě se v~MQTT zprávě pošle hodnota proměnné \texttt{brightness}.
\end{enumerate}

\begin{figure}[htb]
    \centering
    \includegraphics[width=0.3\textwidth]{homeassistant-lovelace-light}
    \caption{%
        Karta \href{https://www.home-assistant.io/lovelace/light/}{Light}
        reprezentující entitu \HAentity{light.alarmclock_ambient} v prostředí
        Lovelace systému Home Assistant
    }
    \label{fig:homeassistant lovelace light}
\end{figure}


\subsubsection{Recorder}
Ve výchozí konfiguraci je přidána integrace \HAintegration{recorder}. Ta
automaticky nahrává historii stavu všech entit, přičemž data jsou ukládána do
SQL databáze. Integrace \HAintegration{history} umožňuje tato historická data
procházet a tvořit z nich grafy. Integrace \HAintegration{logbook} zobrazuje
tato data ve formě textových, chronologicky řazených popisů událostí.

\begin{figure}[htb]
    \centering
    \includegraphics[width=0.5\textwidth]{homeassistant-history}
    \caption{%
        Graf \HAintegration{history} a výpis \HAintegration{logbook} v~detailu
        entity \HAentity{light.alarmclock_ambient}
    }
    \label{fig:homeassistant lovelace light history}
\end{figure}
